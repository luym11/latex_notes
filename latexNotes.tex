\documentclass[11pt]{article}
\topmargin=-.50in
\textheight=9.0in
\textwidth=6.5in
\oddsidemargin=-0.25in
\evensidemargin=-0.25in

\usepackage{times}
\usepackage{amssymb}
\usepackage{amsmath}

\newcommand{\theset}[1]{\left\{#1\right\}}
\newcommand{\magn}[1]{\left| #1 \right|}
\newcommand{\norm}[1]{\left\Vert #1 \right\Vert}

\newcommand{\st}{\textrm{\ \Big\vert\ }}
\newcommand{\tr}{^\mathrm{T}}
\newcommand{\one}{\mathbf{1}}
\newcommand{\zero}{\mathbf{0}}
\newcommand{\expect}[1]{\mathbf{E}\left[#1\right]}
\newcommand{\given}{\text{~}\big\vert\text{~}}
\newcommand{\prob}[1]{\mathbf{Pr}\left[#1\right]}
\newcommand{\amp}{\ \&\ }

\begin{document}

\section*{Document preparation}

\begin{itemize}

\item Never start a sentence with a symbol:
\begin{center}
$F$ denotes the model $\Rightarrow$ The model is denoted by $F$. ---or--- The symbol $F$ denotes the model.
\end{center}
It's ok if a symbol follows a comma as part of a continuation, e.g.,
\begin{center}
In this equation, $f$ denotes the first part, $g$ denotes the second part, and $h$ denotes the last part.
\end{center}

\item Use text font when abbreviating something or writing a word in math mode:
$$x_{opt} \Rightarrow x_\mathrm{opt}$$
$$x_{blue} \Rightarrow x_\mathrm{blue}$$

\item Always use the so-called ``Oxford comma''
\begin{center}
The sizes are large, medium and small. $\Rightarrow$ The sizes are large, medium, and small.
\end{center}
A humorous example of ambiguity is the sentence
\begin{center}
I went to the park with my dogs, Bill and Ted. $\Leftrightarrow$ I went to the park with my dogs, Bill, and Ted.
\end{center}

\item When separating words, use a dash. When separating numbers, use an en-dash. When separating phrases, use an em-dash:
\begin{center}
The multi-agent case is discussed in pages 3--5 of the appendix---if anywhere at all.
\end{center}

\item \LaTeX\  inserts extra space after periods (except after a capital letter). Sometimes, the extra space is unwanted, e.g., after abbreviations:
\begin{itemize}
\item Prof. Lastname
\item Prof.\ Lastname
\end{itemize}
It's subtle, but it's there! Another case is referring to figures, e.g., Fig. 3 vs Fig.\ 3.\\
To avoid the extra space, in \LaTeX\ code either use backslash-space or tilde (instead of space):
\begin{center}
\texttt{Prof.\textbackslash\ Lastname} or 
\texttt{Prof.$\sim$Lastname}
\end{center}
The tilde option will prevent a line break after the period.

\item Regarding bibliographies:
\begin{itemize}
\item Use en-dashes to separate page numbers.
\item The titles of books or conference proceedings are capitalized:
\begin{center}
The book of game theory $\Rightarrow$ The Book of Game Theory
\end{center}
\item The titles of papers are not capitalized, but an exception is the word immediately following a colon:
\begin{center}
The paper of game theory: recent work $\Rightarrow$ The paper of game theory: Recent work
\end{center}
\item To assure capitalization in a bibliography, write the capital letter in between brackets:
\begin{center}
\texttt{A discussion of \{N\}ash equilibrium: \{R\}ecent work}
\end{center}
\item Make sure bibliographies are complete with page numbers, years, etc.

\end{itemize}

\item Refer to sections and figures using capitals:
\begin{center}
Section IV or Figure 3 or Fig.\ 3.
\end{center}

\item Always put parentheses around equation numbers:
\begin{center}
as shown in  (11), or as shown in eq.\ (11)
\end{center}
In the latter example, be sure to avoid the extra space after the period.

\item When using ``respectively'', surround it by commas:
\begin{center}
The beginning and the end, respectively, are short and long.\\
---or---\\
The beginning and the end are short and long, respectively.
\end{center}

\item Avoid saying ``In this paper'' in the abstract.

\item Always specify something after the word ``this'':
\begin{center}
This is an example. $\Rightarrow$ This sentence is an example.
\end{center}

\item If using the word ``we'', try to think of it as ``the readers and the writers'', e.g., in the way a live lecturer could say to the class, ``We will see that this approach is related to....''

\item Never underline anything in text for the sake of emphasis. Rather, use \textit{italics} or \textbf{bold} or \textbf{\textit{bold-italics}} or even \textsc{smallcaps}.

\item \textbf{\textit{Get in the habit}} of looking up answers to questions of style: ``when to use a semi-colon?'', ``where to insert commas for `i.e.' or `e.g.'?'', etc.

\end{itemize}

\end{document}
